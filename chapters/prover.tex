\chapter{NIPoPoW Velvet Fork Prover}
Now that we've established a fork on the Bitcoin Cash chain, we show how our fork can be put to use by creating proofs. We introduce a kind of client called a \emph{prover} which can create all kinds of NIPoPoW proofs on demand.

Let's assume we have an SPV node. We set the bloom filter to our velvet fork tag in order to get all blocks containing only the transactions of the fork. It's possible to then extract from each transaction the payload, which should be the interlink commitment. Since anyone can post such transactions on the blockchain, we have to make sure that the commitment is actually true before we can use it. In order to do that we maintain our own version of the interlink for each block (similar to how the interlinker does) which we know is correct called $\sf realLink$. Then for every block, we compare its commitments (there may be many) with the Merkle Tree root of our $\sf realLink$. If there is a valid commitment we say that the block has a valid interlink. We store the full interlink as $\sf realLink[id(block)]$.

When we talked about NIPoPoWs we mentioned how it's essential that our proof forms a blockchain that can be traversed from start to end. In order to make our proof traversable, whenever we include a block we have to make sure it connects validly to the previous one either by (a) using the regular $\sf previd$ inside the block or (b) using a valid interlink. If we use the $\sf previd$ to link back to the previous block then all the information someone needs to verify the traversability is already there and we don't need to add anything extra. In the case we use the interlink however, we need to provide the Merkle Tree proofs for:

\begin{itemize}
  \item The transaction containing the valid interlink commitment.
  \item The interlink level we use for the connection.
\end{itemize}

If our chain is not traversable the proof is automatically invalid.

We'll now look at the concrete implementation of the prover.

\section{Invalid Interlink Handling}
The original NIPoPoWs paper~\cite{nipopows} gives us some insight into how to handle blocks with invalid interlinks. Let's look at where we need to make changes starting with suffix proofs.

For suffix proofs we need a way to obtain the upchain of a chain, denoted as $\chain\upchain^\mu$. We will now define a procedure to programatically obtain the upchain of a chain called $\sf find \chain\upchain^\mu$.

\subsection{\sf followUp}
$\sf followUp$ takes a block $b$ and a level $\mu$ as parameters. Starting from $b$ it traverses the chain until it reaches another block of level $\mu$ called $B$. In doing so, it is only allowed to use valid pointers. It will only follow a pointer from a block's interlink at level $\mu$ if there is a valid interlink in that block. Otherwise, the only option is to follow the previous block pointer ($\sf previd$). Once $B$ is reached, it is returned alongside the blocks that were traversed as a blockset called $\sf aux$.

\begin{algorithm}[H]
    \caption{\label{alg.nipopow-velvet-follow}followUp
    produces the blocks to connect two superblocks in velvet forks.}
    \begin{algorithmic}[1]
        \Function{\sf followUp}{$B$, $\mu$, realLink, blockById}
            \Let{\textsf{aux}}{\{B\}}
            \While{$B \neq \textsf{Gen}$}
                \If{$B.\textsf{interlink}[\mu] = \textsf{realLink}[\textsf{id}(B)][\mu]$}
                    \Let{id}{B.\textsf{interlink}[\mu]}
                \Else
                    \Comment{Invalid interlink}
                    \Let{id}{B.\textsf{previd}}
                \EndIf
                \Let{B}{\textsf{blockById}[id]}
                \Let{\textsf{aux}}{\textsf{aux}\cup \{B\}}
                \If{$\textit{level}(B) = \mu$}
                    \State\Return{$B$, aux}
                \EndIf
            \EndWhile
            \State\Return{$B$, aux}
        \EndFunction
    \vskip8pt
    \end{algorithmic}
\end{algorithm}


%TODO figure with straight and dropdown cases

\subsection{$\sf find \chain\upchain^\mu$}
Now that we have a way to go back on a level, we can utilize it to construct the entire traversable level $\mu$ up to block $b$, starting from the end of the chain $\chain[-1]$: this is what ${\sf find \chain\upchain^\mu}(b)$ accomplishes. It works by repeatedly calling $\sf followUp$ on the oldest block it has and including the result in its final chain.

\begin{algorithm}[H]
    \caption{\label{alg.nipopow-velvet-upchain}Supplying the necessary data
    to calculate a connected $\chain\upchain^\mu$ during a velvet fork.}
    \begin{algorithmic}[1]
        \Function{{\sf find} $\chain\upchain^\mu$}{$b$, realLink, blockById}
            \Let{B}{\chain[-1]}
            \Let{\textsf{aux}}{\{B\}}
            \Let{\pi}{[~]}
            \If{$\textit{level}(B) \geq \mu$}
                \Let{\pi}{\pi B}
            \EndIf
            \While{$B \neq b$}
                \State $(B, \textsf{aux'})$
                \begin{varwidth}[t]{\linewidth}
                    $\gets$
                    \textsf{followUp}$(B, \mu,
                    \textsf{realLink}, \textsf{blockById})$
                \end{varwidth}
                \Let{\textsf{aux}}{\textsf{aux} \cup \textsf{aux'}}
                \Let{\pi}{\pi B}
            \EndWhile
            \State\Return{$\pi$, aux}
        \EndFunction
    \vskip8pt
    \end{algorithmic}
\end{algorithm}


\section{JavaScript implementation with bcash}
bcash, also used for the interlinker earlier, turns out the be the only option for an SPV node on Bitcoin Cash, making it the obvious choice for the prover.

\subsection{Type Safety}
To ensure code quality we opted to use Flow~\cite{flow}. Flow allowed us to have type safety while keeping all the good characteristics of JavaScript. Unfortunately this introduced a couple of complications.

One is that our source files are not valid JavaScript anymore. In order to run our code we need to pass it through a pre-processor like Babel which emits clean JavaScript that then Node can run.

Another is that all the external libraries we use, including bcash are not written including Flow types. Flow does provide a repository with type definitions for popular packages, however due to the niche category of our dependencies only a few were covered. This meant that the burden was on us to write type definitions for our dependencies. We had to manually write type definitions for bcash and buffer-map. Thankfully, there is previous work~\footnote{Available at \url{https://github.com/OrfeasLitos/TrustIsRisk.js/blob/247c8b182f5bb4bed11df0d3b9136a3e27848798/flow-typed/npm/bcoin_vx.x.x.js}} on type definitions for bcoin which we were able to utilize and extend since the APIs are really similar as bcash is a fork of bcoin.

\subsection{Testing}
Due to the real-time and asynchronous nature of the prover it's important that our implementation is resilient and bug-free. In order to assert that the code has been thoroughly unit tested, boasting a code coverage of >90\%. The tests are written and run with Jest~\cite{jest}. There are also integration tests using real-world data.
