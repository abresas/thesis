\chapter{NIPoPoW Velvet Fork Prover}
\label{chap:prover}
Now that we've established a fork on the Bitcoin Cash chain, we show how our fork can be put to use by creating proofs. We introduce a kind of client called a \emph{prover} which can create all kinds of NIPoPoW proofs on demand.

Let's assume we have an SPV node. We set the bloom filter to our velvet fork tag in order to get all blocks containing only the transactions of the fork. It's possible to then extract from each transaction the payload, which should be the interlink commitment. Since anyone can post such transactions on the blockchain, we have to make sure that the commitment is actually true before we can use it. In order to do that we maintain our own version of the interlink for each block (similar to how the interlinker does) which we know is correct called $\sf realLink$. Then for every block, we compare its commitments (there may be many) with the Merkle Tree root of our $\sf realLink$. If there is a valid commitment we say that the block has a valid interlink. We store the full interlink as $\sf realLink[id(block)]$.

When we talked about NIPoPoWs we mentioned how it's essential that our proof forms a blockchain that can be traversed from start to end. In order to make our proof traversable, whenever we include a block we have to make sure it connects validly to the previous one either by (a) using the regular $\sf previd$ inside the block or (b) using a valid interlink. If we use the $\sf previd$ to link back to the previous block then all the information someone needs to verify the traversability is already there and we don't need to add anything extra. In the case we use the interlink however, we need to provide the Merkle Tree proofs for:

\begin{itemize}
  \item The transaction containing the valid interlink commitment.
  \item The interlink level we use for the connection.
\end{itemize}

If our chain is not traversable the proof is automatically invalid.

We'll now look at the concrete implementation of the prover.

\section{Invalid Interlink Handling}
The original NIPoPoWs paper~\cite{nipopows} gives us some insight into how to handle blocks with invalid interlinks. Let's look at where we need to make changes starting with suffix proofs.

For suffix proofs we need a way to obtain the upchain of a chain, denoted as $\chain\upchain^\mu$. We will now define a procedure to programatically obtain the upchain of a chain called $\sf find \chain\upchain^\mu$.

\subsection{\sf followUp}
$\sf followUp$ takes a block $b$ and a level $\mu$ as parameters. Starting from $b$ it traverses the chain until it reaches another block of level $\mu$ called $B$. In doing so, it is only allowed to use valid pointers. It will only follow a pointer from a block's interlink at level $\mu$ if there is a valid interlink in that block. Otherwise, the only option is to follow the previous block pointer ($\sf previd$). Once $B$ is reached, it is returned alongside the blocks that were traversed as a blockset called $\sf aux$.

\begin{algorithm}[H]
    \caption{\label{alg.nipopow-velvet-follow}followUp
    produces the blocks to connect two superblocks in velvet forks.}
    \begin{algorithmic}[1]
        \Function{\sf followUp}{$B$, $\mu$, realLink, blockById}
            \Let{\textsf{aux}}{\{B\}}
            \While{$B \neq \textsf{Gen}$}
                \If{$B.\textsf{interlink}[\mu] = \textsf{realLink}[\textsf{id}(B)][\mu]$}
                    \Let{id}{B.\textsf{interlink}[\mu]}
                \Else
                    \Comment{Invalid interlink}
                    \Let{id}{B.\textsf{previd}}
                \EndIf
                \Let{B}{\textsf{blockById}[id]}
                \Let{\textsf{aux}}{\textsf{aux}\cup \{B\}}
                \If{$\textit{level}(B) = \mu$}
                    \State\Return{$B$, aux}
                \EndIf
            \EndWhile
            \State\Return{$B$, aux}
        \EndFunction
    \vskip8pt
    \end{algorithmic}
\end{algorithm}


%TODO figure with straight and dropdown cases

\subsection{$\sf find \chain\upchain^\mu$}
Now that we have a way to go back on a level, we can utilize it to construct the entire traversable level $\mu$ up to block $b$, starting from the end of the chain $\chain[-1]$: this is what ${\sf find \chain\upchain^\mu}(b)$ accomplishes. It works by repeatedly calling $\sf followUp$ on the oldest block it has and including the result in its final chain.

\begin{algorithm}[H]
    \caption{\label{alg.nipopow-velvet-upchain}Supplying the necessary data
    to calculate a connected $\chain\upchain^\mu$ during a velvet fork.}
    \begin{algorithmic}[1]
        \Function{{\sf find} $\chain\upchain^\mu$}{$b$, realLink, blockById}
            \Let{B}{\chain[-1]}
            \Let{\textsf{aux}}{\{B\}}
            \Let{\pi}{[~]}
            \If{$\textit{level}(B) \geq \mu$}
                \Let{\pi}{\pi B}
            \EndIf
            \While{$B \neq b$}
                \State $(B, \textsf{aux'})$
                \begin{varwidth}[t]{\linewidth}
                    $\gets$
                    \textsf{followUp}$(B, \mu,
                    \textsf{realLink}, \textsf{blockById})$
                \end{varwidth}
                \Let{\textsf{aux}}{\textsf{aux} \cup \textsf{aux'}}
                \Let{\pi}{\pi B}
            \EndWhile
            \State\Return{$\pi$, aux}
        \EndFunction
    \vskip8pt
    \end{algorithmic}
\end{algorithm}


\section{Velvet Infix Proofs}

\begin{algorithm}[H]
    \caption{\label{alg.nipopow-velvet-infix-go-back}The \textsf{goBack} function which produces the necessary blocks to connect a block $right$ to a preceeding block $left$.}
    \begin{algorithmic}[1]
        \Function{\sf goBack}{$left$, $right$, \textsf{height}, \textsf{realLink}, \textsf{blockById}}
            \State{$aux \gets \emptyset$}
            \While{$right \neq left$}
                \For{$\mu = level(left)$ down to $0$}
                    \If{$\textsf{realLink}[\mathsf{id}(right)][\mu] = right\text{.interlink}[\mu]$}
                        \Let{B'}{\textsf{blockById}[right\text{.interlink}[\mu]]}
                        \If{$\textsf{height}[B'] \ge \textsf{height}[left]$}
                            \Let{aux}{aux \cup \{B'\}}
                            \Let{right}{B'}
                            \State\textbf{break}
                        \EndIf
                    \EndIf
                \EndFor
            \EndWhile
            \State\Return{$aux$}
        \EndFunction
    \vskip8pt
    \end{algorithmic}
\end{algorithm}


\section{JavaScript implementation with bcash}
bcash, also used for the interlinker earlier, turns out the be the only option for an SPV node on Bitcoin Cash, making it the obvious choice for the prover.

\subsection{Overview}
Our prover works as follows. We expect a user to run the prover, passing as a parameter if they want a suffix or infix proof. If they desire an infix proof they have to provide the block of interest. Once they do that the prover runs and syncs with the Bitcoin Cash blockchain via SPV. It does some similar work to the work that an interlinker does: it knows what the correct interlink is for each block. It then checks the included commitments on each block and keeps a valid commitment for each block if found. A block containing a valid commitment is called a \emph{valid velvet}. Interlink pointers can only be used on valid velvets. Once the chain is synced and the prover knows the valid velvets along with the whole blockchain it creates a proof of the requested type and outputs it in JSON format.

\subsection{Interlink Implementation}
Implementing an interlink structure is something we've already covered many times. Our implementation here is based on immutable value objects holding the interlink. Updating an interlink is done in the usual manner and creates a new interlink. What's different here is that we have a \code{proof} method: we use this method to generate a proof of inclusion of a specific pointer in the interlink commitment. We also need a way to be able to get a pointer from a specific level on the interlink and this is accomplished with the \code{at} method. Note that \code{at} can be asked for a level higher than the size of the interlink, in which case the velvet genesis is returned as it's assumed to be a block of level $\infty$.

\begin{lstlisting}[language=Javascript]
class Interlink {
  list: Array<BlockId>;
  genesisId: BlockId;

  constructor(genesisId: BlockId, list: Array<BlockId> = []) {
    this.genesisId = genesisId;
    this.list = list;
  }

  update(blockId: BlockId) {
    const list = this.list.slice();
    const lvl = level(blockId);
    for (let i = 0; i <= lvl; ++i) {
      if (i < list.length) list[i] = blockId;
      else list.push(blockId);
    }
    return new Interlink(this.genesisId, list);
  }

  proof(level: number) {
    return merkle.createBranch(hash256, level, [...this.list]);
  }

  hash() {
    return merkle.createRoot(hash256, [...this.list])[0];
  }

  get length() {
    return this.list.length;
  }

  at(lvl: number) {
    if (lvl >= this.list.length) {
      return this.genesisId;
    }
    return this.list[lvl];
  }
}
\end{lstlisting}

\subsection{Locating Velvet Transactions}
Locating the velvet transactions is very easy considering that all our velvet transactions are SPV tagged. The only thing we have to do is add the SPV tag in our bloom filter. We do this right before connecting to the network.

\begin{lstlisting}[language=Javascript]
module.exports = class ProverNode extends bcash.SPVNode {
  // ...
  async connect() {
    this.pool.watch(Buffer.from(VELVET_FORK_MARKER));
    await super.connect();
  }
  // ...
}
\end{lstlisting}

\subsection{Extracting Interlink Commitments}
we have to check that the transaction the valid form described earlier (\code{OP\_RETURN <tag> <commitment>}). In order to do that we look at the outputs of each transaction and only keep any outputs of our form. If some output matches then we extract the script's third element which is the commitment.

\begin{lstlisting}[language=Javascript]
const extractInterlinkHashes = compose(
  map(
    compose(
      head,
      drop(1)
    )
  ),
  filter(isInterlinkData),
  map(
    compose(
      getCleanScriptData,
      prop("script")
    )
  ),
  filter(isBurnOutput),
  prop("outputs")
);
\end{lstlisting}

\subsection{Handling Merkle Blocks}
When a block or transaction of a block matches our bloom filter, the block is relayed by full nodes as a \code{merkleblock} message. MerkleBlocks contain the header of the actual block as well as the transactions ids that matched. It also contains a proof of inclusion for all the transactions ids provided. One can check this proof of inclusion with accordance to the \code{merkleRoot} inside the block header. The transactions are included in subsequent \code{tx} messages. bcash hides this process from the user: on a new block event a MerkleBlock is provided and it contains a \code{txs} record, where the transactions that matched are included in full.

In order to check if we have matched transactions in a block all we have to do is check if its \code{.txs} array is non-empty. If it is not then this block is definitely not a valid velvet. If it is not, then we have to ensure a correct form and extract the commitment, also saving it. We utilize the method \code{extractInterlinkHashes} and map it over all transactions in the block.

\begin{lstlisting}[language=Javascript]
const extractInterlinkHashesFromMerkleBlock = compose(
  unnest,
  map(extractInterlinkHashes),
  prop("txs")
);
\end{lstlisting}

\subsection{RealLink \& Verifying Interlink Commitments}
We extend the \code{realLink} discussed earlier. Other than allowing us to get the interlink of any block it also allows us to check if a block is a valid velvet. It is able to do that as it gets notified for every new block seen during the synchronisation. When a new block appears, it gets informed about the block id and the interlinks included in the block. It keeps a running interlink in the same fashion the interlinker does so because the blocks events happen in-order it can check whether its running interlink matches the ones included in the block. In case there's a match the block is marked as a valid velvet and the running interlink is updated so that RealLink can be ready to process the next block.

\begin{lstlisting}[language=Javascript]
module.exports = class RealLink {
  blockIdToInterlink: BufferMap;
  runningInterlink: Interlink;
  validBlocks: BufferSet;

  constructor(genesisId: BlockId) {
    this.blockIdToInterlink = new BufferMap();
    this.runningInterlink = new Interlink(genesisId);
    this.validBlocks = new BufferSet();
  }

  onBlock(newBlockId: BlockId, interlinks: Array<Buffer>) {
    if (this.blockIdToInterlink.has(newBlockId)) {
      return;
    }

    this.blockIdToInterlink.set(newBlockId, this.runningInterlink);
    if (
      interlinks.some(interlink =>
        interlink.equals(this.runningInterlink.hash())
      )
    ) {
      this.validBlocks.add(newBlockId);
    }
    this.runningInterlink = this.runningInterlink.update(newBlockId);
  }

  get(blockId: BlockId) {
    return this.blockIdToInterlink.get(blockId);
  }

  hasValidInterlink(blockId: BlockId) {
    return this.validBlocks.has(blockId);
  }
};
\end{lstlisting}

\subsection{Handling Merkle Blocks Redux}
All information regarding the chain is sorted in a \code{VelvetChain} (TODO change name). A \code{VelvetChain} is the main entrypoint for block events. It is the single source of truth about the blockchain. When there is a new block, the whole block is associated with its id for later lookup, and its height is also recorded. It then gathers the block's interlink commitments and forwards them as a new event to \code{realLink}.

\begin{lstlisting}[language=Javascript]
  // ...
  onBlock(blk: bcash.MerkleBlock, height: number) {
    const id = blk.hash();
    if (this.blockById.has(id)) {
      return;
    }

    ++this.height;

    if (!this.genesis) {
      this.genesis = id;
      this._realLink = new RealLink(this.genesis);
      console.log("genesis was %O", blk);
    }

    this.blockById.set(id, blk);
    this.heightById.set(id, this.height);
    this.blockList.push(blk.hash());

    const includedInterlinkHashes = extractInterlinkHashesFromMerkleBlock(blk);
    this.realLink.onBlock(id, includedInterlinkHashes);

    this.lastBlock = id;
  }
  // ...
\end{lstlisting}

\subsection{Following Up}
We implement the \code{followUp} algorithm as described in Algorithm~\ref{alg.nipopow-velvet-follow}. \code{followUp} acts on a \code{VelvetChain} and requires a \code{newerBlockId} which is equivalent to $B$ and $\mu$.

We also see use of the \code{levelledPrev} function. \code{levelledPrev}, given a block id and a level will attempt to cross the interlink pointer at the requested level. If the block is not a velvet block it will fail and just return the \code{previd}.

A departure from the exact algorithm is that in our implementation we don't return a block set. Instead we return a block list which is sorted by height. By convention we also say that a block $A$ is \textit{left} of another block $B$ if $B.\mathsf{height} > A.\mathsf{height}$.

\begin{lstlisting}[language=Javascript]
  followUp(newerBlockId: BlockId, mu: Level): Array<BlockId> {
    const genesis = nullthrows(this.genesis);
    let id = newerBlockId;
    let path = [id];
    while (!id.equals(genesis)) {
      id = this.levelledPrev(id, mu);
      path.push(id);

      if (level(id) === mu) {
        break;
      }
    }

    path.reverse();
    return path;
  }
\end{lstlisting}

\subsection{Finding the Velvet Upchain}
Utilizing the \code{followUp} algorithm to implement Algorithm~\ref{alg.nipopow-velvet-upchain} is also pretty straightforward. However instead of only $\chain \upchain^\mu$ we would like to implement $\chain\upchain^\mu\{B:\}$ as this is needed by the prover. Programmatically we reference $B$ by its id as  \code{leftBlockId}. The original algorithm repeatedly calls to \code{followUp} until it has reached the genesis, which in code can be optimized away as we know that we only need blocks after (and including) $B$. This is why we \code{followUp} until $B$ only, and take care of the special case where $B$ is not a $\mu$ superblock. We look for the path from \code{followUp} for any blocks which are left of $B$ in the chain. If that is the case, the algorithm only keeps the part of the path which is right of $B$, and $B$ if it is included in the path and then exits.

\begin{lstlisting}[language=Javascript]
  findVelvetUpchain(
    mu: Level,
    leftBlockId: BlockId,
    rightBlockId: ?BlockId = this.lastBlock
  ): {
    muSubchain: Array<BlockId>,
    wholePath: Array<BlockId>
  } {
    const genesis = nullthrows(this.genesis);
    let id = nullthrows(rightBlockId);

    let wholePath = [];
    let muSubchain = [];
    if (level(id) >= mu) {
      muSubchain.push(id);
    }

    let goneTooFar = false;
    while (!id.equals(leftBlockId) && !id.equals(genesis) && !goneTooFar) {
      let path = this.followUp(id, mu);
      let outOfRangePathIndex = _.findLastIndex(path, x =>
        this.isLR(x, leftBlockId)
      );
      if (outOfRangePathIndex !== -1) {
        path.splice(0, outOfRangePathIndex + 1);
        goneTooFar = true;
      }

      id = path[0];

      if (level(id) >= mu) {
        muSubchain.push(id);
      }
      wholePath = path.slice(1).concat(wholePath);
    }

    wholePath = [id, ...wholePath];

    muSubchain.reverse();
    return {
      muSubchain,
      wholePath
    };
  }
\end{lstlisting}

\subsection{Crafting Suffix Proofs}
Using these primitives we can now move on to creating our first proofs. The interface implemented allows us to directly translate the pseudocode shown in Algorithm~\ref{alg.nipopow-velvet-prover}. The code will first initialize the two parts of the proof, \code{pi} and \code{chi} to be empty. Then it will pick $max\mu$ to be the level of the rightmost stable block, $\chain[-k]$, with $B$ initially set to the genesis block, $G$. Then for each level from $max\mu$ up to and including 0, it utilizes \code{findVelvetUpchain} to get the velvet upchain from the rightmost stable block up to and including $B$. If the $\mu$ superblocks of the chain are more than $m$, the path returned is included and $B$ is set to the $m$-th rightmost $\mu$ superblock of the path.

\begin{lstlisting}[language=Javascript]
function suffixProof({
  chain: C,
  k,
  m
}: {
  chain: VelvetChain,
  k: number,
  m: number
}): Array<BlockId> {
  let leftId = nullthrows(C.genesis);
  let pi: Array<BlockId> = [],
    chi: Array<BlockId> = [];
  let rightMostStableId = C.idAt(-k - 1);
  let maxMu = C.interlinkSizeOf(rightMostStableId);

  for (let mu = maxMu; mu >= 0; --mu) {
    let { muSubchain, wholePath } = C.findVelvetUpchain(
      mu,
      leftId,
      rightMostStableId
    );
    let newBlocks = wholePath;
    if (mu > 0) {
      if (muSubchain.length <= m) {
        continue;
      }
      leftId = muSubchain[muSubchain.length - m];
      let leftIdInWholePath = newBlocks.findIndex(id => id.equals(leftId));
      newBlocks = newBlocks.slice(0, leftIdInWholePath);
    }
    pi = pi.concat(newBlocks);
  }

  for (let i = -k; i < 0; ++i) {
    chi.push(C.idAt(i));
  }

  return pi.concat(chi);
}
\end{lstlisting}

\subsection{Following Down}
%  -----      -----
%            +
%           +
%          +
As we noted in Section~\ref{ssec:infix}, a proof is a valid infix proof with regards to a set of blocks if and only if it contains those blocks, and it is traversable. This means th a suffix proof is also an infix proof for any possible subset of its included blocks (assuming the infix predicate requires only block header information which is included for every block).

If however, an infix proof for a specific block is required, it should be possible to augment our suffix proof with it. To do this, we implement \code{followDown}, as seen in Algorithm~\ref{alg.nipopow-infix-follow}. 

The purpose of the algorithm is to link two blocks, starting from the higher-level block on the right, \code{hi} and ending at the lower-level block on the left, \code{lo}. It attempts to take the largest possible steps, using the maximum-level pointer available on \code{hi}'s valid interlink which does not surpass the block of interest. If the interlink on some block the algorithm ends up is invalid, the block's \textsf{prevId} is used instead. Continuing in this manner, a traversable path is formed between \code{hi} and \code{lo}.

% TODO: improvement where we can use partially valid interlinks - we don't have this facility in code as this is a recent development, should this be included?

\begin{lstlisting}[language=Javascript]
function followDown(C: VelvetChain, hi: BlockId, lo: BlockId) {
  let B = hi;
  let aux = [];
  let mu = level(hi);
  assert(C.heightOf(hi) >= C.heightOf(lo));
  while (!B.equals(lo)) {
    let Bp = C.levelledPrev(B, mu);
    if (C.heightOf(Bp) < C.heightOf(lo)) {
      --mu;
    } else {
      if (!B.equals(hi)) {
        aux = [B, ...aux];
      }
      B = Bp;
    }
  }
  return aux;
}
\end{lstlisting}

Following down is not enough to promote an existing suffix proof to a traversable infix proof as we will see now.

\subsection{Going Back}
%  -----A     C----
%       |    /
%       |   /
%       \++B

Suppose we have a suffix proof and we select two blocks, $A$ and $C$. These blocks are selected so that they are adjacent in the proof and that $B$, our block of interest, is contained between them in the underlying blockchain. Also, $C$ should be right of $A$ to avoid ambiguity. By following down we obtain a traversable path from $C$ to $B$. If we augment a suffix proof with this path, the result may not be traversable, as we have no guarantee that there is a valid path from $B$ to $A$. In order for our final proof to be traversable we have to ensure a valid path between those two blocks.

To this end we implement \code{goBack}, which mimics \code{followDown} and connects \code{lo} to \code{hi} in the same manner.

% TODO: goBack can be deprecated in favor of flip followDown

\begin{lstlisting}[language=Javascript]
function goBack(C: VelvetChain, lo: BlockId, hi: BlockId) {
  let aux = [];
  assert(C.heightOf(hi) <= C.heightOf(lo));
  while (!lo.equals(hi)) {
    for (let mu = level(hi); mu >= 0; --mu) {
      const b = C.levelledPrev(lo, mu);
      if (C.heightOf(b) >= C.heightOf(hi)) {
        lo = b;
        aux.unshift(lo);
        break;
      }
    }
  }
  aux.shift();
  return aux;
}
\end{lstlisting}

\subsection{Crafting Infix Proofs}
Using these tools we are able to craft our infix proofs. The implementation works only for a single block of interest, $B$, but can be easily extended to multiple such blocks. It is assumed that the infix predicate depends only on block header data. The algorithm starts by creating a standard suffix proof. In case the suffix proof already contains the block of interest the suffix proof is enough and can be returned. Otherwise, it will find the blocks in the proof before and after $B$: $A$ and $C$. Then the proof returned will be a concatenation of the following:

\begin{itemize}
  \item The suffix proof up to and including $A$.
  \item The path obtained via \code{goBack}, ordered from $A$ to $B$.
  \item $B$.
  \item The path obtained via \code{followDown}, ordered from $B$ to $C$.
  \item The rest of the suffix proof starting from $C$.
\end{itemize}

\begin{lstlisting}[language=Javascript]
function infixProof({
  chain: C,
  blockOfInterest: B,
  k,
  m
}: {
  chain: VelvetChain,
  blockOfInterest: BlockId,
  k: number,
  m: number
}): Array<BlockId> {
  const suffixP = suffixProof({ chain: C, k, m });
  if (suffixP.some(x => x.equals(B))) {
    return suffixP;
  }

  const pi = suffixP.slice(0, -k),
    chi = suffixP.slice(-k);

  const afterBIndex = _.findIndex(pi, e => C.heightOf(e) >= C.heightOf(B));
  const afterBBlock = pi[afterBIndex];
  const beforeBIndex = Math.max(afterBIndex - 1, 0);
  const beforeBBlock = pi[beforeBIndex];

  return [
    ...pi.slice(0, afterBIndex),
    ...goBack(C, B, beforeBBlock),
    B,
    ...followDown(C, afterBBlock, B),
    ...pi.slice(afterBIndex),
    ...chi
  ];
}
\end{lstlisting}

\subsection{Type Safety}
To ensure code quality we opted to use Flow~\cite{flow}. Flow allowed us to have type safety while keeping all the good characteristics of JavaScript. Unfortunately this introduced a couple of complications.

One is that our source files are not valid JavaScript anymore. In order to run our code we need to pass it through a pre-processor like Babel which emits clean JavaScript that then Node can run.

Another is that all the external libraries we use, including bcash are not written including Flow types. Flow does provide a repository with type definitions for popular packages, however due to the niche category of our dependencies only a few were covered. This meant that the burden was on us to write type definitions for our dependencies. We had to manually write type definitions for bcash and buffer-map. Thankfully, there is previous work~\footnote{Available at \url{https://github.com/OrfeasLitos/TrustIsRisk.js/blob/247c8b182f5bb4bed11df0d3b9136a3e27848798/flow-typed/npm/bcoin_vx.x.x.js}} on type definitions for bcoin which we were able to utilize and extend since the APIs are really similar as bcash is a fork of bcoin.

\subsection{Testing}
Due to the real-time and asynchronous nature of the prover it's important that our implementation is resilient and bug-free. In order to assert that the code has been thoroughly unit tested, boasting a code coverage of $90\%$ as shown in Figure~\ref{fig:test-coverage}. The tests are written and run with Jest~\cite{jest}. There are also integration tests using real-world data.

\pngfigure{test-coverage}{Snapshot from the test coverage report for the prover codebase.}
