\chapter{Conclusion}

\section{Summary}
In this work we have studied the state-of-the-art constructions for creating proofs in the Prover-Verifier Model. We adapted the most fit construction for the purpose, NIPoPoWs, for the Bitcoin setting, by utilizing the User Activated Velvet Fork deployment strategy outlined in the NIPoPoWs paper and sidestepping the constant difficulty requirement.


We designed and implemented the specifics of the Velvet Fork, including methods for encoding the interlink. We paid close attention to the consumption of the interlinks, making them easily consumable by lite nodes by utilizing SPV tagged outputs.

We provided two implementations for the \textbf{interlinker}, an implementation which requires a full-node and one which is based on a lite-node with SPV. These implementations actually create the velvet fork and can be easily adapted to fork on any Bitcoin-based cryptocurrency, as well as perform any other kind of velvet fork. They are both released as open-source and are production-grade software.

We deployed the interlinker implementations on the Bitcoin Cash testnet, closely monitoring their performance and reliability, while fixing any anomalies due to bugs in our software as they developed. 

We provided cost estimates for deploying our User-Activated Velvet Fork on Bitcoin Cash mainnet, where it can be seen that even though one needs to spend one transaction per block, due to the low fees on Bitcoin Cash is incredibly manageable even for individuals.

We analyzed blockchain data from Bitcoin and Bitcoin Cash, on their mainnets and testnets and discovered that the interlink encoding could be improved. We provided an optimization for interlink encoding based on our observations called \textbf{interlink compression}. We provided statistics which indicate how much of an improvement such a change can bring on real-world blocks.

We provided an implementation for the \textbf{prover}, which is based on a lite node. The prover is able to interact in the Bitcoin network, and has knowledge of the velvet fork, as well as of the actual interlink of each block. It is capable of discovering the velvet transactions and recovering the interlink commitments from them, and knowing if a block has a valid interlink or not. It is capable of creating both suffix and infix proofs for Bitcoin Cash.

\section{Proof Consumption}
The velvet fork and prover together enable any user to create NIPoPoW proofs on Bitcoin Cash. These proofs can be utilized as part of many applications. We will now look at some potential applications of very high interest to the cryptocurrency community.

\subsection{Super-light Clients}
Currently lite clients operate by accepting multiple SPV proofs. As we have seen these proofs are linear in chain size. One can very easily imagine modified light clients which have the ability to verify and compare NIPoPoW proofs instead of SPV proofs in order to discover the longest chain or whether a transaction is included in the longest chain.

\subsection{Cross-chain Transactions}
Another popular application is cross-chain transactions: for example sending an amount of Bitcoin to Ethereum trustlessly. Currently when one wishes to move from one cryptocurrency to the other they usually utilize a centralized exchange. This is both time-consuming and trustless.

There is also the issue of the denomination: one would usually exchange their Bitcoin for an equivalent amount of Ethereum, according to the market rate. While in some cases this is desired, it is usually the case that one desires to transfer their funds to another cryptocurrency but keep them in the original denomination.

To solve this issue there are tokens in the Ethereum network like WBTC \cite{wbtc} which claim to offer an ERC20 token pegged 1:1 to Bitcoin, but have the need for a trusted third party called a custodian who is entrusted to generate tokens at will. It should be obvious that such solutions are not trustless.

An alternative would be to offer such a token but have the contract issue an amount of tokens if and only if it can can be sufficiently proven that the actor who requested the issuance has burned the same amount of Bitcoin on the other chain. Many constructions of this sort have been explored in the literature \cite{xclaim}, including most importantly a construction using NIPoPoWs by Kiayias and Zindros \cite{pow-sidechains}. An implementation of this construction in Solidity for the Ethereum chain has been provided by Christoglou \cite{christoglou}.

\section{Future Work}
\subsection{Bitcoin Cash mainnet Deployment}
In order to make our work really useful to the general public the obvious step would be to deploy our interlinker on the mainnet network of Bitcoin Cash. This is a straightforward process, and it is directly supported by the software as is. We note that due to the low fees of Bitcoin Cash and consequently the low cost for operating an interlinker, it is very accessible for anyone to operate such a fork.

\subsection{Interlink Compression Implementation}
While we propose interlink compression, the development was very recent and has not been yet integrated in our interlinkers and prover. This change can be implemented and deployed in a backwards-compatible manner, meaning that the existing interlinks will not be rendered invalid. The changes to the interlinker implementation are fairly straightforward: only the interlink encoding would have to change. However, the prover has to support both the block list and block set constructions, and prefer the block set interlinks where available.

\subsection{Implementation and Deployment For Other Major Cryptocurrencies}
The usefulness of NIPoPoWs is not limited to Bitcoin Cash only. The software can be easily deployed on any Bitcoin-based cryptocurrency, including Bitcoin itself, only requiring minimal changes.

Other than Bitcoin-based cryptocurrencies, NIPoPoWs work for any Proof-of-Work cryptocurrency. We think velvet forks like the one outlined in this work would be viable and incredibly beneficial for many other cryptocurrencies, most notably Ethereum, Ethereum Classic, and Dash.

\subsection{Verifier Smart Contract Implementation}
The Proof-of-Work sidechains implementation by Christoglou \cite{christoglou} outlined earlier is unable to verify proofs created from velvet forked chains like the ones our provers create. It is therefore necessary for the implementation to be modified in order to support such proofs. Such a contract, if deployed on the Ethereum chain would allow users to trustlessly transfer Bitcoin Cash to the Ethereum chain by making use of proofs our prover can generate for the Bitcoin Cash chain.
