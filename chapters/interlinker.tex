\chapter{Velvet Fork Implementation}
Now that we have seen how NIPoPoWs work and what options we have in our disposal to deploy them to existing blockchains, we will investigate how we implemented NIPoPoWs on Bitcoin Cash.

\pdffigure{blocks-with-their-interlinks}{Example of blocks with their corresponding interlinks for $T = 100000$. Note that for the upcoming block marked with \textbf{?} we can produce its interlink without knowing its hash.}

Since Bitcoin Cash blocks don't contain the interlink we have to utilize a velvet fork. We could choose a regular velvet fork. A regular velvet fork would require for a miner minority to exist and run code which would add the interlink inside the coinbase transaction. Finding and contacting miners would require a lot of non-technical work and persuasion. It would also mean that probably only a minority of the blocks would be validly interlinked. Seeing that persuading miners would require a lot of work and would not bring an ideal outcome we decided to implement a User-Activated Velvet Fork instead.

To this end, we need to make sure that a transaction is included in every single block containing its implied interlink. We do this by implementing a service which for every new block, calculates the expected interlink of the upcoming block and sends a transaction including this interlink in hopes that it will be included in the upcoming block. If this is indeed achieved then that block will indeed contain its valid interlink. We henceforth call the service which does this an \emph{interlinker}. An example of blocks and their corresponding interlinks can be seen on Figure~\ref{fig:blocks-with-their-interlinks}.

\section{Picking a Velvet Genesis}
Naturally one would expect the interlink to start from the real blockchain genesis as it would enable proofs for any already existing block of that blockchain. However, for older blocks there can be no improvement. There is no other option than providing the full chain as a proof since no older blocks contain any interlinks. Thus in order to avoid accounting in our interlink for blocks in the past that can only be connected to the real genesis by supplying the full chain we choose a new genesis called the \emph{velvet genesis}. For our purposes in Bitcoin Cash testnet we chose the block with height 1257603 as our velvet genesis.

\section{Interlink encoding}
We now have to consider our options for representing the interlink. 
Bitcoin Cash uses SHA-256 for the block hashes, meaning that each block id consists of 32 bytes. 
A naive encoding would be one where each 32-byte block hash from level $0$ up to $\infty$ is concatenated, with the $\infty$ pointer always pointing to the genesis block..
For 35 levels, which according to Figure~\ref{fig:bitcoin-cash-testnet-levels-after-velvet-genesis} is a reasonable figure, this encoding would take approximately 1.12KB.
Considering that \code{OP\_RETURN} scripts are limited in size to 223 bytes, this would only allow us to include up to 6 pointers at best.

\pdffigure{bitcoin-cash-testnet-levels-after-velvet-genesis}{Number of levels assuming our selected velvet genesis.}

Putting this limitation aside, the fee of the transaction is proportional to the transaction size, and since we're going to be sending a transaction for every block (which is mined approximately every 10 minutes), it is important that the fee is minimized.

Thus in order to save space, we only include a commitment to the interlink in our transactions. Specifically, we take the Merkle Tree root of the Merkle Tree with leafs the block hashes starting from level $0$ up to $\infty$. This way, our interlink encoding is constant size and we can easily provide compact proofs for any of the levels.

\section{Discoverability}
We've talked about how just including the interlink somewhere on a block is what really matters but it is crucial that we make this information easy to discover. We achieve this in two ways. First, we include the interlink commitment inside a special \code{OP\_RETURN} output. Such outputs are already being used for storing arbitrary data in blocks~\cite{arbitrary-data} therefore we adopt this method for storing our interlinks. Second, we aim to make this interlink discoverable for lite nodes, so we don't require our users to download a whole block in order to look into it. We achieve this by utilizing a method called \emph{SPV tagged outputs}~\cite{spv-tagged}.

SPV tagged outputs are outputs that are tagged so that they can be discovered by SPV nodes who add the ``tag'' to the filter. Specifically, we form outputs of the form \code{OP\_RETURN baba deadbeef}, where \code{baba} is our tag and \code{deadbeef} is our payload (in our case, the interlink commitment). A bloom filter for \code{baba} will then match this output and subsequently the transaction that contains it and this is how our specialized SPV nodes can discover our outputs. The full nodes forwarding the velvet transactions to the SPV nodes don't have any knowledge of what a velvet fork even is, let alone that they are forwarding its transactions.

The tag we use for our transactions is \code{696e7465726c696e6b}, which is the ASCII encoding of \code{interlink}. An example of such a real-world velvet transaction created by our deployed interlinker on the Bitcoin Cash testnet can be seen on Figure~\ref{fig:actual-velvet-tx}.

\pngfigure{actual-velvet-tx}{An actual velvet fork transaction.}

\section{Fault Tolerance}
It is important to note that the interlinker works on a best-effort basis. Due to the nature of Velvet Forks though, no failure is fatal. The types of failures are as follows:

\begin{itemize}
  \item \textbf{Crash failure}: The interlinker crashes or halts.
  \item \textbf{Omission failure}: The interlinker fails to push a transaction upon seeing a new block.
  \item \textbf{Timing failure}: The interlinker pushes a transaction which fails to be included in the upcoming block.
  \item \textbf{Response failure}: The interlinker pushes a transaction with an invalid interlink.
  \item \textbf{Arbitrary failure}: The interlinker pushes a transaction before a new block is seen or pushes a duplicate for a specific upcoming block twice.
\end{itemize}

We will study how these failures can be mitigated in the next section.

\section{Viability}
Making the operation of the interlinker affordable is key in order to allow many parties to run it. We will estimate the cost of operation now.

Our transaction size ($\sf txBytes$) is constant at exactly 244 bytes. The median Bitcoin Cash fee per byte ($\sf feePerByte$) at the time of writing (November 2018) is 1 satoshi. Multiplied by our transaction size this gives us a transaction fee ($\sf txFee = txBytes * feePerByte$) of 244 satoshis.  Our complete estimations based on the current price of Bitcoin Cash can be seen on Table~\ref{tbl:cost-analysis}.

\begin{table}
  \centering
  \begin{tabular}{|c|c|c|}
    \hline
    Cost per & Formula & Current estimation \\
    \hline
    Day & $\sf txFee * 10 * 24$ & 0.13€ \\
    Month & $\sf perDay * 30$ & 3.9€ \\
    Year & $\sf perMonth * 12$ & 46.8€ \\
    \hline
  \end{tabular}
  \caption{Cost analysis of operating an interlinker on the Bitcoin Cash mainnet.}
  \label{tbl:cost-analysis}
\end{table}

We provide two implementations of an interlinker which both run in production. We'll now look at the pros and cons of each.

\section{Bitcoin-ABC}
Our first implementation is built on top of Bitcoin-ABC. Bitcoin-ABC is the reference implementation for Bitcoin Cash in C++. It is a fork of the original Bitcoin codebase (now Bitcoin Core), and it was the first ever implementation to support Bitcoin Cash. It has a very active community of developers and users. Due to its heritage from Bitcoin Core the code is very well written and tested, and any new feature for the Bitcoin Cash chain appears on Bitcoin-ABC first.

We run Bitcoin-ABC as a full node and interface with it using JSON-RPC. The interlinker is a Python module which knows (a) the location and credentials to connect with the full node and (b) the velvet fork genesis block id. We use the python-bitcoinrpc library~\cite{python-bitcoinrpc} for the JSON-RPC communication.

After the interlinker makes sure the full node is fully synced it starts waiting for new blocks. As described above we need our interlinker to get notified whenever there's a new block so that it can send a transaction with the new interlink for inclusion in the upcoming block. There's two options to get notified for new blocks from a full node.

\subsection{Block Discovery}

The first option is to utilize the ZeroMQ~\cite{zmq} functionality of Bitcoin-ABC. If compiled with the appropriate flags, Bitcoin-ABC can create a ZeroMQ channel where it will send notifications for new blocks and transactions. While the method seems very modern, it has some pitfalls. The main pitfall is race conditions: it is possible that the node pushes out a ZeroMQ notification about a block, however that block has not finished saving in the node's database, causing an immediate \code{getblock} request on the block to fail. Also, the ZeroMQ functionality is not enabled by default: one needs to compile the node with specific flags. While both issues are not fatal, in order to keep the interlinker as compatible with existing nodes as possible and to avoid workarounds we decided against using this functionality.

What we ended up using is busy polling through JSON-RPC with the \code{getbestblockhash} method. Every 5 seconds the interlinker will check the best block hash and if it doesn't match the one previously given then this means that there's been a new block.

\begin{lstlisting}[language=Python]
tip_id = ''
while True:
    possibly_new_tip_id = rpc.getbestblockhash()
    if possibly_new_tip_id == tip_id:
        sleep(NEW_TIP_CHECK_INTERVAL_SECONDS)
        continue
    tip_id = possibly_new_tip_id
\end{lstlisting}

\subsection{Reaction to a New Block}

In case there's a new block, the interlinker will promptly compute the correct interlink for it, by updating the interlink with all the intermediate blocks from the velvet fork genesis up to and including the new block. It will then compute the Merkle Tree root and include it in an SPV tagged output. It will then wrap the output inside a change transaction and send it to the network.

We also see the use of \code{shelve} on the code below. \code{shelve} is a Python library which allows us to persist Python objects on disk and access them on demand. We will see promptly how we utilize \code{shelve} to act as a database of computed interlinks in order to speed up our computations.

\begin{lstlisting}[language=Python]
    logger.info('new block "%s"', tip_id)
    with shelve.open(db_path) as db:
        new_interlink = interlink(tip_id, db)
    logger.debug('new interlink "%s"', new_interlink)
    logger.info('mtr hash "%s"', new_interlink.hash().hex())
    logger.info('velvet tx "%s"', send_velvet_tx(new_interlink.hash()))
\end{lstlisting}

\subsection{Calculating the Interlink}

The naive way to calculate the interlink would be starting from the velvet genesis block to keep a running interlink and sequentially update it. Considering that between the tip and the velvet genesis there could be millions of blocks, and for each block we need to issue a JSON-RPC call to get its block hash by using \code{getblockhash}. We experimentally found this method to take upwards of 15 seconds for approximately 15000 blocks, averaging at 1000 blocks/second.

To make this process faster we cache all interlinks we compute. This caching happens inside the \code{store} shown in the code below. \code{store} is a dictionary which maps block ids to their corresponding interlinks. In order to compute an interlink for a given tip we traverse the blockchain from the tip until we end up at a block for which the interlink is already available. We always ensure the cache contains at least the interlink for the velvet genesis which is empty and only contains the genesis pointer (of level $\infty$). We keep the block ids between the tip and the block with the cached interlink and then proceed to update the interlink that was found with each intermediate block id, in order. In doing so we also save each interlink we compute to its corresponding block id in the store, so that it will be available if we ever need it again. After the interlink has been updated with all blocks we are done and we can return it as the final result.

\begin{lstlisting}[language=Python]
def interlink(tip_id, store=None):
    store = {} if store is None else store
    intermediate_block_ids = deque()
    intermediate_id = tip_id
    if VELVET_FORK_GENESIS not in store:
        store[VELVET_FORK_GENESIS] = genesis_interlink()
    while intermediate_id not in store:
        intermediate_block_ids.appendleft(intermediate_id)
        intermediate_id = prev(intermediate_id)

    intermediate_interlink = store[intermediate_id]
    for block_id in intermediate_block_ids:
        intermediate_interlink = store[block_id] = \
                intermediate_interlink.update(block_id, level(block_id))
    return intermediate_interlink
\end{lstlisting}

\subsection{Interlink}

It's time to look at how we programmatically represent the interlink. We take the simplest approach which is to represent it as a POPO (Plain Old Python Object).

To instantiate an interlink one needs a velvet genesis id and optionally a list of blocks. This list of blocks represents the interlink pointers, where the element at $block[i]$ is the id of the most recent block at level $i$. If the list of blocks is not provided the interlink is initially empty.

\begin{lstlisting}[language=Python]
class Interlink:
    def __init__(self, genesis, blocks=None):
        self.genesis = normalized_block_id(genesis)
        self.blocks = [] if blocks is None else blocks
\end{lstlisting}

The \code{update} method is a straightforward adaptation of Algorithm~\ref{alg.nipopow-interlink}. It's important to note that instead of mutating the \code{self} object, \code{update} operates on a new copy which it then returns. This makes any \code{Interlink} object in essence immutable which allows us to cache it very elegantly.

\begin{lstlisting}[language=Python]
    def update(self, block_id, level):
        block_id = normalized_block_id(block_id)
        blocks = self.blocks.copy()
        for i in range(0, level+1):
            if i < len(blocks):
                blocks[i] = block_id
            else:
                blocks.append(block_id)
        return Interlink(genesis=self.genesis, blocks=blocks)
\end{lstlisting}

The commitment is generated by the \code{hash} method defined on \code{Interlink}, which after getting the representation of the interlink as an array, including the velvet genesis as the last element, then calculates the Merkle Tree root (abbreviated as \code{mtr}) of a tree with the elements of this array as leaves.

\begin{lstlisting}[language=Python]
    def as_array(self):
        return self.blocks + [self.genesis]

    def hash(self):
        return mtr(self.as_array())
\end{lstlisting}

\subsection{Bitcoin-style Merkle Trees}

For implementing the \code{mtr} function seen above we searched for existing Python libraries offering this functionality. While some libraries came up, none of them implemented the Bitcoin-style Merkle Trees we were after. This functionality is definitely implemented in some of more general Python Bitcoin libraries we make use of however it isn't exposed. This led us to write our own implementation of Bitcoin-style Merkle Trees.

We only implement the root calculation as described in Section~\ref{sec:merkle-trees} in \code{mtr}. \code{mtr} accepts an array of leafs. The function works by taking a level of the tree (initially the leafs) and repeatedly generating the upper level. This process continues until the level reached only has one element, where it's declared the root and is returned as the result. The function for generating the upper level, \code{next\_level} uses many Python idioms and this requires an explanation. It starts by taking a level and then pairing its element with its adjacent. This happens with \code{zip\_longest} which takes 2 lists and "zips" them together like so: \code{zip\_longest([1, 2, 3], [4, 5, 6]) = [(1, 4), (2, 5), (3, 6)]}. For the uninitiated we provide a Haskell implementation below.

\begin{lstlisting}[language=Haskell]
zip_longest :: [a] -> [a] -> a -> [a]
zip_longest (x:xs) (y:ys) fillvalue = (x, y) : zip_longest xs ys fillvalue
zip_longest (x:xs) [] fillvalue = (x, fillvalue) : zip_longest xs [] fillvalue
zip_longest [] (y:ys) fillvalue = (fillvalue, y) : zip_longest [] ys fillvalue
zip_longest [] [] _ = []
\end{lstlisting}

The two lists passed to \code{zip\_longest} are \code{level[::2]} which is all even elements of \code{level} and \code{level[1::2]} which is all odd elements. We use \code{zip\_longest} instead of the regular \code{zip} to account for the case where \code{level} is of odd length and we need to hash the last element with itself.

\begin{lstlisting}[language=Python]
    from itertools import zip_longest
    def next_level(level):
        return [hash_siblings(l, r) for l, r in zip_longest(level[::2], level[1::2], fillvalue=level[-1])]
\end{lstlisting}

After getting the list of pairs we go through it hashing the pairs together.

\begin{lstlisting}[language=Python]
    from hashlib import sha256
    def hash_siblings(l, r):
        h1, h2 = sha256(), sha256()
        h1.update(l)
        h1.update(r)
        h2.update(h1.digest())
        return h2.digest()
\end{lstlisting}

This setup makes our Merkle Tree root calculation very straightforward.

\begin{lstlisting}[language=Python]
def mtr(leafs):
    assert len(leafs) > 0
    level = leafs
    while len(level) > 1:
        level = next_level(level)
    return level[0]
\end{lstlisting}

\subsection{Velvet Transactions}

For creating the SPV tagged outputs we utilize the python-bitcoinlib library~\cite{python-bitcoinlib}. We construct the outputs array including a single script with the \code{OP\_RETURN}, our SPV tag which is the ASCII encoding for \code{interlink} and our payload which we expect to be our interlink commitment. We then serialize a transaction with this output.

\begin{lstlisting}[language=Python]
from bitcoin.core import CMutableTxOut, CScript, CMutableTransaction, OP_RETURN
def create_raw_velvet_tx(payload_buf):
    VELVET_FORK_MARKER = b'interlink'
    digest_outs = [CMutableTxOut(0, CScript([OP_RETURN, VELVET_FORK_MARKER, payload_buf]))]
    tx = CMutableTransaction([], digest_outs)
    return tx.serialize().hex()
\end{lstlisting}

The next stage is to actually send the transaction. Of course a transaction with only one output is incomplete and invalid. In order to fill it in with the appropriate inputs and change output we turn to our Bitcoin-ABC node. We use the \code{fundrawtransaction} JSON-RPC method in order to do all the above. We then only have to sign and send the transaction out in the network.

\begin{lstlisting}[language=Python]
def send_velvet_tx(payload_buf):
    change_address = rpc.getaccountaddress("")
    funded_raw_tx = rpc.fundrawtransaction(create_raw_velvet_tx(payload_buf),
            {'changeAddress': change_address})['hex']
    signed_funded_raw_tx = rpc.signrawtransaction(funded_raw_tx)['hex']
    return rpc.sendrawtransaction(signed_funded_raw_tx)
\end{lstlisting}

\subsection{Funding}

It is important to note here that in order to send the transactions, the interlinker has to pay transaction fees as seen earlier. However, we haven't talked about the interlinker controlling a wallet or private keys which would make it seem like there is no way to fund the transactions. Because of how the JSON-RPC methods work we don't need to specify a private key or wallet external to the full node and we can let the full node create and pay for our transactions using its default wallet. Thus the way to fund the interlinker is to fund the default wallet of the full node.

\subsection{Testing}

The code is thoroughly unit-tested using pytest~\cite{pytest}. The Interlink implementation is a very crucial part of our implementation and is tested as follows.

\begin{lstlisting}[language=Python]
from interlink import Interlink

def test_interlink():
    interlink = Interlink(genesis=b'\x01')
    interlink = interlink.update('0c', 5)
    interlink = interlink.update('0b', 4)
    interlink = interlink.update('0a', 2)
    assert interlink.as_array() == [b'\x0a', b'\x0a', b'\x0a', b'\x0b', b'\x0b', b'\x0c', b'\x01']

def test_interlink_cascading():
    interlink = Interlink(genesis=b'\x02')
    interlink = interlink.update('0a', 2)
    interlink = interlink.update('0b', 4)
    interlink = interlink.update('0c', 5)
    assert interlink.as_array() == [b'\x0c'] * 6 + [b'\x02']

def test_interlink_str():
    assert str(Interlink(genesis=b'\x02')) == '[1 * 02]'
\end{lstlisting}

The Bitcoin-style Merkle Trees are also very important to get right. To this end we generate testcases adapted from actual Bitcoin blocks by making sure the merkle tree root of their transaction ids matches the one found in the block. We also make sure to also test the boundaries (for example having 0 and 1 leafs).

\begin{lstlisting}[language=Python]
import pytest
import json

from merkle import mtr

def do_test_fixture(fixture_path):
    with open(fixture_path, 'r') as f:
        fixture = json.load(f)
    expected_merkleroot = bytes.fromhex(fixture['merkleroot'])[::-1]
    txs = [bytes.fromhex(tx_id)[::-1] for tx_id in fixture['txs']]
    assert mtr(txs) == expected_merkleroot

def test_bitcoin_block_100():
    do_test_fixture('./fixtures/merkle/bitcoin_block_100.json')

def test_bitcoin_cash_testnet_2_txs():
    do_test_fixture('./fixtures/merkle/bitcoin_cash_testnet_2_txs.json')

def test_bitcoin_cash_testnet_big():
    do_test_fixture('./fixtures/merkle/bitcoin_cash_testnet_1259110.json')

def test_no_leafs():
    with pytest.raises(AssertionError):
        mtr([])
\end{lstlisting}

\section{bcash}
The second implementation is build on top of the bcash JavaScript library. bcash implements a full node and wallet functionality exclusively in JavaScript without being based on the Bitcoin C++ codebase as did most previous solutions. A major advantage of bcash is that it implements an SPV node too, which is very useful since the interlinker only needs the block ids of the chain but doesn't care about the block contents, so requiring a full node to run it would be a waste of space and bandwidth.

\section{Deployment}
At the time of writing of this thesis, both interlinkers run in production on the Bitcoin Cash testnet chain and have been for over 2 months. We started with our first deployment late September 2018, with the first prototype of our Python interlinker. This gave us a chance to catch bugs that only appear on long running processes and make sure our software is resilient to network and other library failures. As new features and bug fixes kept rolling in our software we continuously updated the deployments.

docker images
docker-compose

\section{Experimental Data}
\pdffigure{testnet-deployment-reliability}{Reliability of our testnet deployment.}

We will now examine real-world data gathered from our Bitcoin Cash testnet deployment. On Figure~\ref{fig:testnet-deployment-reliability} we can look at the reliability of our interlinker deployment. We can see that about 60\% of the blocks have been correctly interlinked with our velvet transactions, and about 10\% of them only have 1 interlink pointer missing, which makes them potentially usable. With $\infty$ we denote the case where all pointers are missing due to a block only having incorrect interlinks or no interlinks at all.

\pdffigure{bitcoin-cash-testnet-levels}{Level distribution on Bitcoin Cash testnet.}

On Figure~\ref{fig:bitcoin-cash-testnet-levels} we see the level distribution on the Bitcoin Cash testnet. As expected, the level growth is logarithmic compared to the chain length.
