\chapter{Bitcoin Cash Velvet Fork Implementation}
Since Bitcoin Cash blocks don't contain the interlink we have to utilize a User-Activated Velvet Fork. To this end, we need to make sure that a transaction is included in every single block containing its implied interlink. We do this by implementing a service which for every new block, calculates the expected interlink of the upcoming block and sends a transaction including this interlink in hopes that it will be included in the upcoming block. If this is indeed achieved then that block will indeed contain its valid interlink. We henceforth call the service which does this an \emph{interlinker}.

\section{Interlink encoding}
Including the whole interlink inside the transaction would make the transaction really big. Assuming a naive encoding of the interlink, where each 256-bit block hash from level $0$ up to $\infty$ is concatenated, on the Bitcoin Cash testnet at the time of writing just the interlink would be ~0.6875KB. In Bitcoin the fee of the transaction is proportional to the transaction size, and since we're going to be sending a transaction for every block (which is mined approximately every 10 minutes), it's important that we keep this cost down. In order to save space, we only include a commitment to the interlink in our transactions. Specifically, we take the Merkle Tree root of the Merkle Tree with leafs the block hashes starting from level $0$ up to $\infty$. This way, our interlink encoding is constant size and we can easily provide compact proofs for any of the levels.

\section{Discoverability}
We've talked about how just including the interlink somewhere on a block is what really matters but it's crucial that we make this information easy to discover. We achieve this in two ways. First, we include the interlink commitment inside a special \code{OP\_RETURN} output. Such outputs are already being used for storing arbitrary data in blocks so it's where someone would expect our interlink to be. Second, we aim to make this interlink discoverable for lite nodes, so we don't require our users to download a whole block in order to look into it. We achieve this by utilizing a method called \emph{SPV tagged outputs}. 

SPV tagged outputs are outputs that are tagged so that they can be discovered by SPV nodes who add the "tag" to the filter. Specifically, we form outputs of the form \code{OP\_RETURN baba deadbeef}, where \code{baba} is our tag and \code{deadbeef} is our payload (in our case, the interlink commitment). A bloom filter for \code{baba} will then match this output and subsequently the transaction that contains it and this is how our specialized SPV nodes can discover our outputs. To our knowledge this is the first time such outputs are used in Bitcoin. We make extensive use of this feature as we'll see on the next chapter.

\section{Viability}
The cost of running an interlinker on the Bitcoin Cash chain is estimated to 10€ per month at the time of writing.

We provide two implementations of an interlinker. Since the writing of this thesis, both interlinkers run in production on the Bitcoin Cash testnet chain. We'll now look at the pros and cons of each.

\section{Bitcoin-ABC}
Our first implementation is built on top of Bitcoin-ABC. Bitcoin-ABC is the reference implementation for Bitcoin Cash in C++. It is a fork of the original Bitcoin codebase (now Bitcoin Core), and it was the first ever implementation for Bitcoin Cash. It has a very active community of both developers and users. The code is very well written and tested, and any new feature for the Bitcoin Cash chain appears on Bitcoin-ABC first.

We run Bitcoin-ABC as a full node and interface with it using JSON-RPC. The interlinker is a Python module which knows (a) the credentials to connect to the full node and (b) the velvet fork genesis block id.

After it makes sure the full node is fully synced it starts waiting for new blocks. As described above we need our interlinker to get notified whenever there's a new block so that it can send a transaction with the new interlink for inclusion in the upcoming block. There's two options to get notified for new blocks from a full node.

The first option is to utilize the ZeroMQ functionality of Bitcoin-ABC. If compiled with the appropriate flags, Bitcoin-ABC can create a ZeroMQ channel where it will send notifications for new blocks and transactions. While the method seemed very modern we decided against using it. The main reason was race conditions: it's possible that the node pushes out a ZeroMQ notification about a block, however that block is still not stored correctly yet in the block db, causing an immediate \code{getblock} request on the new block to fail. While this could be worked around, this combined with the fact that the software needs to be recompiled for this option to work made it seem like it was not the best option available.

What we ended up using is busy polling through JSON-RPC, using the \code{getbestblockhash} method. Every 5 second, or any defined interval in the configuration, the interlinker will check the best block hash and if it doesn't match the one previously given then this means that there's been a new block.

In case there's a new block, the interlinker will promptly compute the correct interlink for it, by updating the interlink with all the intermediate blocks from the velvet fork genesis up until and including the new block. It will then compute the Merkle Tree root and included it in an SPV tagged output, and then create and send a transaction with that output.

It's important to note here that in order to send the transactions, the interlinker has to spend money. However, we haven't talked about the interlinker having a wallet or private keys and any way to fund it. Because of how the JSON-RPC methods work we don't need to specify a private key or wallet external to the full node and we can let the full node create our transactions using its default wallet. Thus the way to fund the interlinker is to fund the default wallet of the full node.

\section{bcash}
