\documentclass[11pt]{llncs}
\usepackage{preamble}

\begin{document}
\title{Constructing interoperable blockchains using NIPoPoWs}
\date{\today}
\author{Kostis Karantias\\
    \email{cse32454@cse.uoi.gr}}
\institute{University of Ioannina}
\maketitle
\noindent
\makebox[\linewidth]{\small \today}

\newpage

\begin{abstract}
  In this paper we examine the real-world applications of the recent blockchain
  primitive called Non Interactive Proofs of Proof of Work (NIPoPoWs). We
  enable the use of the primitive on Bitcoin Cash by implementing the first
  known velvet fork on a blockchain. We provide concrete implementations for
  creating a velvet fork, and generating proofs on a blockchain that has been
  forked in this way.
\end{abstract}

\newpage

\tableofcontents

\newpage

\thispagestyle{plain}

\section{Introduction}

This is the introduction.

\subsection{Motivation}
\subsection{Objectives}

\section{Background}

\subsection{Bitcoin}

\subsubsection{History}
\subsubsection{Block Structure}
\subsubsection{Proof of Work}
\subsubsection{Merkle Trees}
\subsubsection{Simple Payment Verification}
\subsubsection{Bitcoin Cash}

\subsection{Non-Interactive Proofs of Proofs of Work}

\subsubsection{Sublinear SPV}
\subsubsection{Interlink}
\subsubsection{Proof Creation}
\subsubsection{Proof Verification}
\subsubsection{Velvet Forks}
\subsubsection{User-Activated Velvet Forks}

\section{Bitcoin Cash Velvet Fork Implementation}

\subsection{Bitcoin-ABC}
\subsection{Interlinker}

\section{NIPoPoW Velvet Fork Prover Implementation}

\subsection{bcash}
\subsection{Testing}

\bibliography{bibliography}

\end{document}
