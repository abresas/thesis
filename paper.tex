\documentclass[11pt]{llncs}
\usepackage{preamble}

\begin{document}
\title{Constructing interoperable blockchains using NIPoPoWs}
\date{\today}
\author{Kostis Karantias\\
    \email{cse32454@cse.uoi.gr}}
\institute{University of Ioannina}
\maketitle
\noindent
\makebox[\linewidth]{\small \today}

\newpage

\begin{abstract}
  In this paper we examine the real-world applications of the recent blockchain
  primitive called Non Interactive Proofs of Proof of Work (NIPoPoWs). We
  enable the use of the primitive on Bitcoin Cash by implementing the first
  known velvet fork on a blockchain. We provide concrete implementations for
  creating a velvet fork, and generating proofs on a blockchain that has been
  forked in this way.
\end{abstract}

\newpage

\tableofcontents

\newpage

\thispagestyle{plain}

\section{Introduction}

\subsection{Related Work}
\subsection{Objectives}
\subsection{Structure}

\section{Background}

\subsection{Bitcoin}

\subsubsection{History}
Bitcoin~\cite{bitcoin} was introduced in 2008 by Satoshi Nakamoto. TODO

\subsubsection{Block Structure}
\subsubsection{Proof of Work}
\subsubsection{Merkle Trees}
A merkle tree~\cite{merkle} is a cryptographic primitive. TODO

\subsubsection{Simple Payment Verification}
\subsubsection{Bitcoin Cash}
In 2017 Bitcoin faced severe scalability issues~\cite{onscaling}. Its limited
1MB block size meant that it could only support a maximum of 7 transactions per
second. As Bitcoin's popularity had exploded at the time, the problem was
hugely exacerbated. The most prominently proposed solution for this was a block
size increase, however no consensus was reached. The discussions ended with a
fork of the main Bitcoin chain which allowed for 8MB blocks, called Bitcoin
Cash.

\subsection{Non-Interactive Proofs of Proofs of Work}

\subsubsection{Sublinear SPV}
There have been previous attempts to create proofs smaller in size than SPV
proofs~\cite{KLS}. NIPoPoWs is the first secure construction~\cite{nipopows}.
TODO

\subsubsection{Interlink}
The idea behind interlinks comes in 2012~\cite{highway}. TODO

\subsubsection{Proof Verification}
\subsubsection{Suffix Proofs}
\subsubsection{Infix Proofs}
\subsubsection{Velvet Forks}
Velvet forks~\cite{nipopows,velvet} describe a formalization of adding
arbitrary data inside blocks in order to allow potential applications without
sacrificing the backwards compatibility of the blockchain. TODO

\subsubsection{User-Activated Velvet Forks}

\section{Bitcoin Cash Velvet Fork Implementation}

\subsection{Bitcoin-ABC}
\subsection{Interlinker}

\section{NIPoPoW Velvet Fork Prover Implementation}

\subsection{bcash}
\subsection{Testing}

\section{Results \& Future Work}

\bibliography{bibliography}

\end{document}
